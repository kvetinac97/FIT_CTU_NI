\section{Úvod}

\vspace{12pt}

\subsection{Organizace předmětu}

\subsubsection*{Zápočet}

\begin{itemize}
    \item 2 kvízy v MARASTu po 6 bodech \textit{(nutno splnit)}
    \item 10 minikvízů po 1 bodu \textit{(min. 7 bodů)}
    \item vesměs programovací domácí úkol za 6 bodů \textit{(min. 1 bod)}
\end{itemize}

\subsubsection*{Zkouška}

\begin{itemize}
    \item zkoušková písemka za max 40 bodů \textit{(2x 20 bodů)}
    \item minimum 50 \% bodů z první části a celkově
    \item ústní zkouška za max 40 bodů \textit{(právo veta)}
    \item dvě otázky, na které se půjde písemně připravit
\end{itemize}

Kvízy budou 5. a 10. týden semestru, minikvízy od 2. týdne, úkol po 6. týdnu semestru, měkký termín před Vánocemi.

\begin{figure}[H]
    \centering
    \begin{equation}
        \mathbb{N} = \{0, 1, 2, \ldots, \infty\}
    \end{equation}
    V NI-MPI bude do přirozených čísel patřit i nula.
\end{figure}

\subsection{Témata}

\subsubsection*{Vícerozměrné funkce a optimalizace}

Mnoho problémů lze formulovat jako optimalizační problémy, kde minimalizujeme funkci určující zisk, vzdálenost nebo třeba dobu běhu algoritmu.

Pokud je tato funkce \textit{(která může být více proměnných)} zadaná analyticky, dá se optimum hledat.

\subsubsection*{Strojová čísla a numerika}

Spojitá matematika na počítači a stabilita numerických algoritmů. Jak probíhá ukládání čísel a intervalů čísel v počítači a jak lze odhadovat chyby, které zde vznikají. Z této části bude onen \textit{(programovací)} úkol.

\subsubsection*{Obecná algebra -- grupy, tělesa}

Konečné grupy a tělesa jsou zdrojem nástrojů pro kryptografii, hashovací funkce, generování náhodných čísel a další\ldots
