\section{Funkce více proměnných}

\vspace{12pt}

\subsection{Vícerozměrný prostor a funkce}

\subsubsection{Norma a vzdálenost}

\textbf{Norma} na vektorovém prostoru V je zobrazení $||\cdot|| : V \to \mathbb{R}_0^+$ splňující, že norma nulového vektoru je 0, lze z ní vytknout skalár $\alpha$ a platí zde trojúhelníková nerovnost. \textbf{Vzdálenost} vektorů $x, y \in V$ pak definujeme jako $d(x,y) = ||x - y||$.

\vspace{4pt} \noindent Platí zde tedy triviálně $d(x,y) = 0 \Leftrightarrow x = y$, symetrie a trojúhelníková nerovnost.

\vspace{4pt} \noindent Obecně, pro libovolné $p \geq 1$ je
\begin{equation}
    ||x||_p = \sqrt[p]{\sum_{i=1}^n{|x_i|^p}}
\end{equation}
norma (pro $p = 2$ jde o euklidovskou normu, $p = 1$ součtová norma).

\subsubsection{Okolí bodu}

Pokud volím normu, pak $\delta$-okolí bodu $x$ je množina
\begin{equation}
    H_\delta(x) = \{b \in \mathbb{R}^n: ||x - b|| < \delta\}
\end{equation}
takzvaná \textbf{otevřená koule} o středu $x$ a poloměru $\delta$. Obecně pro všechna okolí píšeme jednoduše $H(x)$.

\vspace{4pt} \noindent O $x \in \mathbb{R}^n$ řekneme, že je \textbf{hromadným bodem} $M$, pokud pro všechna $r > 0$ platí
\begin{equation}
    H_r(x) \setminus \{x\} \cap M \neq \emptyset
\end{equation}
Bod $M$, který není hromadný, se nazývá \textbf{izolovaný}.

\subsubsection{Limita posloupnosti a funkce}

O posloupnosti $(x_i)_{i=0}^{\infty}$ řekneme, že má \textbf{limitu} $L \in \mathbb{R}^n$, pokud
\begin{equation}
    (\forall \epsilon > 0)(\exists N)(\forall n > N)(x_n \in H_{\epsilon}(L))
\end{equation}

\noindent Reálná funkce více proměnných je zobrazení $f: D_f \to \mathbb{R}$, kde $D_f \subseteq \mathbb{R}^n$ (pro kladné celé n), opět zde platí definiční obor a obor hodnot.

\vspace{4pt} \noindent \textbf{Graf} funkce $f$ je množina bodů z funkce.

\vspace{4pt} \noindent Řekneme, že \textbf{funkce} $f: D_f \to \mathbb{R}$ má \textbf{limitu} $L \in \mathbb{R}$ v hromadném bodě $b$ množiny $D_f$ pokud

\begin{equation}
    \forall H(L) \quad \exists H(b) \quad x \in (D_f \cap H(b)) \setminus \{b\} \Longrightarrow f(x) \in H(L)
\end{equation}

\subsubsection{Spojitost funkce}

Řekneme, že funkce $f: D_f \to \mathbb{R}, D_f \subset \mathbb{R}^n$ je \textbf{spojitá} v bodě $x_0 \in D_f$, pokud:

\begin{equation}
    \forall \epsilon > 0 \quad \exists \delta > 0 \quad x \in (D_f \cap H_\epsilon(x_0)) \Longrightarrow f(x) \in H_\epsilon(f(x_0))
\end{equation}

\noindent V izolovaném bodě je každá funkce spojitá.

\subsubsection{Extrémy funkce}

O funkci $f$ řekneme, že má v bodě $b \in D_f$:

\begin{itemize}
    \item \textbf{lokální minimum}, pokud $\exists \delta > 0, \forall x \in (D_f \cap H_\delta(b)), f(x) \geq f(b)$;
    \item \textbf{ostré lokální minimum}, pokud $\exists \delta > 0, \forall x \in (D_f \cap H_\delta(b)), f(x) > f(b)$;
    \item \textbf{globální minimum}, pokud $\forall x \in D_f, f(x) \geq f(b)$.
\end{itemize}

\noindent Pokud máme $D_f \subset \mathbb{R}^n$, která je \textbf{omezená} (je podmnožinou nějaké otevřené koule) a \textbf{uzavřená} (obsahuje i svou hranici -- body, jejichž každé okolí obsahuje bod z $D_f$ i bod mimo $D_f$), pak má spojitá funkce $D_f \to \mathbb{R}$ v $D_f$ globální minimum/maximum.

\subsection{Parciální derivace}

\subsubsection{Definice}

\textbf{Parciální derivace} funkce $f$ve směru osy $x_i$ v bodě $b = (b_1, b_2, \ldots, b_n) \in D_f$ takovém, že $\exists H(b) \subset D_f$, je
\begin{equation}
    \lim_{h \to 0}\frac{f(b_1, b_2, \ldots, b_i + h, \ldots, b_n) - f(b_1, b_2, \ldots, b_i, \ldots, b_n)}{h} = L,
\end{equation}
pokud tato limita existuje.

\vspace{4pt} \noindent Značíme $\frac{\partial f}{\partial x_i}(b) = L$, jedná se o sm. tečny ke grafu funkce $f$ ve směru osy $x_i$.

\subsubsection{Gradient funkce}

\textbf{Gradient funkce} $f$ v bodě $b \in D_f$ je vektor
\begin{equation}
    \nabla f(b) = \left(\frac{\partial f}{\partial x_1}(b), \frac{\partial f}{\partial x_2}(b), \ldots, \frac{\partial f}{\partial x_n}(b)\right).
\end{equation}

Představuje směr nejvyššího růstu funkce $f$ (kde je "nejstrmější").

\subsubsection{Derivace ve směru}

Derivace funkce $f$ ve směru $v \in \mathbb{R}^{n,1} = \mathbb{R}^n, ||v|| = 1$ v bodě $b \in D_f$ takovém, že $\exists H(b) \subset D_f$, je
\begin{equation}
    \nabla_v f(b) = \lim_{h \to 0}\frac{f(b + hv) - f(b)}{h}
\end{equation}

\noindent Platí, že pokud jsou všechny parciální derivace $f$ na nějakém okolí bodu $b$ spojité, pak
\begin{equation}
    \nabla_v f(b) = \nabla f(b) \cdot v.
\end{equation}

\subsubsection{Tečná nadrovina}

Jedná se o jakési sjednocení tečen ve všech směrech v daném bodě. Její rovnicí je pak
\begin{equation}
    z = \frac{\partial f}{\partial x_1}(b)(x_1 - b_1) + \frac{\partial f}{\partial x_2}(b)(x_2 - b_2) + \ldots + \frac{\partial f}{\partial x_n}(b)(x_n - b_n) + f(b)
\end{equation}

\subsubsection{Nutná podmínka lokálního extrému}

Nechť má funkce $f: D_f \to \mathbb{R}, D_f \subset \mathbb{R}^n$, v bodě $b$ parciální derivaci podle i-té proměnné.

\vspace{4pt} \noindent Pokud $f$ má v bodě $b$ lokální extrém, pak
\begin{equation}
    \frac{\partial f}{\partial x_i}(b) = 0.
\end{equation}

\subsubsection{Stacionární bod}

Pokud existuje gradient funkce $f$ v bodě $b$, pak existence lokální extrému implikuje $\nabla f(b) = 0$.

\vspace{4pt} \noindent Body $b \in D_f$ splňující $\nabla f(b) = 0$ se nazývají \textbf{stacionární}.

\subsubsection{Kritický bod}

Kritickým bodem (podezřelým z extrému) je ten, kde je gradient nulový (stacionární bod) nebo gradient neexistuje.
