\section{Experimentální práce s algoritmy}

\subsection{Účely}

\textbf{Teorie}: srovnání se známými algoritmy, s optimem.

\noindent \textbf{Aplikace}: vhodnost algoritmu pro zamýšlenou funkci.

\vspace{4pt}

\noindent \textbf{Složitosti}:

\begin{itemize}
    \item nejhorší případ (vzácný a nezajímavý, složitá analýza)
    \item průměrný případ (proveditelné pro jednoduché případy)
\end{itemize}

\noindent $\to$ experimentální vyhodnocení = složitost jednotlivých kroků nelze spočítat.

\vspace{4pt}

\subsubsection*{Postup}

Otázka co chci zjistit

$\to$ plán experimentu

$\to$ provedení experimentu

$\to$ interpretace výsledků

$\to$ odpověď \textit{(toto běželo x hodin, proto\ldots)}

\subsection{Metriky}

Zajímají nás metriky vstupu a výstupu a jejich závislost.

Algoritmus: heuristika => problémově závislé části algoritmu => parametry algoritmu

\noindent $\to$ v souvislosti s algoritmem se zkoumá počet navštívených konfigurací

\vspace{4pt}

\noindent Implementace: datové struktury -- kódování -- platforma

$\to$ v souvislosti s implementací se zkoumá doba běhu (čas CPU)

\vspace{4pt}

Při měření závislostí metrik někde můžeme držet stejné metriky vstupu, u~některých to ale nejde (únava topiče).

U SATu se zjistilo, že vstupní metrikou je poměr počtu klauzulí k počtu proměnných.

Při generování sady instancí musím pro každou instanci se \textbf{zadanou metri\-kou} nechat \textbf{stejnou} pravděpodobnost. Pokud instance sbírám, snažím se získat co nejvíce instancí (se stejnou metrikou).

Toto proženu algoritmem, změřím data, proženu to statistikou, abych se zbavil variance a interpretuji.

\subsection{Statistika pro 1 hodnotu}

Nejlepší je použít průměr nebo medián, použít vhodné parametry rozložení ($\mu$, $\sigma$) a následně kvalitně reprezentovat.

\subsection{Srovnání algoritmů}

Počet kroků algoritmu nechť je náhodná proměnná, pokud má A lepší parametry, je lepší. Pokud pro každou instanci A je lepší, pak A dominuje.

\subsection{Randomizovaný algoritmus}

Všechny hodnoty generátoru náhodných čísel jsou stejně pravděpodobné + potlačení variance z randomizace.

Spolehlivá data můžeme ověřit pravděpodobnostním mechanismem testování hypotéz.

\subsection*{Ale...}

Někdy nejsme schopni charakterizovat problém $\to$ nemáme věrohodný generátor a nemůžeme potlačit varianci. Pak můžeme použít standardní sady problémů k určité úloze, kde najdeme i možnost porovnání s již existujícími algoritmy. Tomuto se říká \textbf{inženýrská algoritmika} -- srovnávat lze pouze vzhledem k dané sadě instancí.

\subsection{Interpretace experimentu}

\textbf{Data}: chování na konkrétních instancích s konkrétními parametry (pohled zvenčí, mnoho jednotlivých dat)

\vspace{4pt}
\noindent \textbf{Interpretace}: obecný závěr -- musíme provést extrapolaci, bývá kvalitativní, pohled zevnitř, jednoduchá formulace

\subsection{Prezentace experimentu}

Při prezentaci čtenáře zajímá důležitost otázky, relevance experimentu, reprodukovatelnost jeho důvěra v interpretaci a přijetí odpovědi.

Autor tedy musí znát cílovou skupinu a předat ji data, získat její důvěru a přesvědčit ji. K tomu slouží tzv. IMRaD.

\subsection{IMRaD}

IMRaD se skládá z:

\begin{itemize}
    \item Introduction: je potřeba X, široce zaměřené
    \item Methods \& Results: úžeji zaměřené, obsahuje reporty
    \item Discussion (\& Conclusion): udělali jsme X, široce zaměřené
\end{itemize}

\subsubsection{Introduction}

Proč byla studie provedena? Jaký byl účel výzkumu? Jaká byla otázka? Jakou hypotézu jsme testovali?

$\to$ buduje výchozí bod pro další výklad

\subsubsection{Methods}

Kdy, kde a jak jsme experimenty provedli? Proč jsme použili dané metody? Jak jsme ověřili korektnost? Jaký jsme použili experimentální materiál a proč?

\subsubsection{Results}

K jakým jsme došli výsledkům? Jakou jsme získali odpověď? Potvrdili jsme hypotézu?

$\to$ používáme zde vždy \textbf{pouze} všechna relevantní data

\subsubsection{Discussion}

Co odpověď může znamenat? Jaký má vztah k dosavadnímu výzkumu? Jaké má perspektivy?

$\to$ vše, co podporuje i nepodporuje tvrzení článku, ale ne zbytečnosti

\subsubsection{Zbytečnosti}

Tabulky a grafy musí být vždy dohromady, jedno bez druhého nefunguje dobře. Chybějící sloupce v tabulkách jsou také na škodu. Grafy musí obsahovat srozumitelné sdělení. Popisky musí být výstižné. Pozor na pravdivosti tvrzení!

\subsection{Pro KOPy}

Cílovou skupinu známe, zadání také. Důležité je předat důvěru a přesvědčení ve výsledky + že autor rozumí tomu, co dělá.