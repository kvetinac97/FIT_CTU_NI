\section{Stavový prostor}

U \textbf{dynamického} programování zakládám řešení vždy na \textbf{optimálním} řešení problému nižšího stupně.

\textbf{Batoh}: Platí, že následník každé konfigurace má větší celkovou váhu a cenu, následník \textbf{nepřípustné} konfigurace je nepřípustná konfigurace.

\subsection{Formální definice}
Nechť $X = {x_1, x_2, \ldots, x_n}$ jsou \textbf{konfigurační proměnné} problému $\Pi$. Nechť $Z = {z_1, z_2, \ldots, z_m}$ jsou \textbf{vnitřní proměnné} algoritmu A řešícího instanci I problému $\Pi$. Pak \textbf{každé ohodnocení} konfiguračních a vnitřních proměnných je \textbf{stav} algoritmu A řešícího I.

Nechť $S = {s_i}$ je množina všech stavů algoritmu A řešícího I a $Q = {q_j}$ množina \textbf{operátorů} $S \to S$ takových, že $q_j(s_i) \neq s_i$ pro všechny kombinace (vždycky to nějak změní). Pak dvojici (S,Q) nazveme \textbf{stavovým prostorem} algoritmu A řešícího I.

\textbf{Akce}: aplikace jednoho konkrétního operátoru $q_j$ na jeden konkrétní stav $s_i$. Stavový prostor se dá namapovat na \textbf{orientovaný graf} $H=(S,E)$, kde hrana $(s_i, s_j)$ odpovídá jedné akci, vrcholy jsou jednotlivé stavy.

\textbf{Okolí stavu} je množina stavů, do kterých se dostanu \textbf{jedním} krokem, aplikací jedné operace. Stavy z okolí se nazývají \textbf{sousední} stavy.

\textbf{k-okolí stavu} s je množina stavů, dosažitelných s aplikací nejméně jedné a nejvýše \textbf{k} operací/kroků. Výměny uvnitř grafu nejsou užitečné sama o sobě. Pouze inverzní operátory taky nestačí.

\subsection{Grafy stavového prostoru}

\begin{itemize}
    \item \textbf{acyklický}: omezená délka cesty, jednoduché řízení heuristiky, \textit{hladové algoritmy}
    \item \textbf{cyklický}: vyžaduje komplikovanější řízení, pokročilé heuristiky
\end{itemize}

Pro každé dva stavy nazveme délku nejkratší cesty z $s_1$ do $s_2$ v grafu H \textbf{vzdáleností} uzlu $s_2$ od $s_1$. Aby byla heuristika \textbf{pokročilá}, musí platit:

\begin{itemize}
    \item \textbf{dostupnost}: mezi každými dvěma uzly musí být cesta $\to$ graf musí být silně souvislý
    \item \textbf{symetrie}: pro každé dva stavy musí být jejich vzdálenost $s_1$ do $s_2$ a $s_2$ do $s_1$ přibližně \textbf{stejná}
\end{itemize}

\textbf{Konfigurace} je obecně \textbf{podgraf} grafu G v závislosti na algoritmu. \textbf{Uzel stavového grafu} je ten daný stav, hranou je operace, pozor na příklad v úkolech souvisejících s grafy.

\textbf{POZOR !!!} Některé úlohy mají za úkol najít \textbf{cestu} ke stavu jiného objektu \textit{(hra Sokoban)}. Množina stavů je pak \textbf{množina posloupností akcí}, operace nad stavem \textbf{přidej na konec}. Něco jiného je pak \textbf{stav scény} a \textbf{stav algoritmu}.

\subsection{Strategie pohybu stavovým prostorem:}

\subsubsection{Úplná strategie}

Úplná strategie navštíví \textbf{všechny} stavy, kromě těch, o kterých víme, že nedávají optimální řešení \textit{(otevírá cestu k prořezávání)}

\textbf{Úplný algoritmus} dokáže odpovědět, že instance nemá řešení. Použití úplné strategie v lokální heuristice dává úplný algoritmus \textit{(za cenu složitosti)}.

\subsubsection{Systematická strategie}

Systematická strategie je úplná strategie, která navštíví každý stav \textbf{nejvýše jednou}. Typicky systematické strategie obsahují proměnné:

\begin{itemize}
    \item \textbf{open}: neprozkoumaní sousedé \textit{(fronta -- BFS, zásobník -- DFS, prioritní fronta -- best first)}
    \item \textbf{closed}: prozkoumané stavy
    \item \textbf{state}: aktuální stav
    \item \textbf{best}: nejlepší nalezený stav
\end{itemize}

Pokud neuvažujeme \textbf{prořezávání}, je nejhorší případ roven \textbf{hrubé síle} a tento případ nastane v případě \textbf{neexistujícího} řešení. Tyto strategie naleznou \textit{(optimální)} řešení, existuje-li.

\subsection{Lokální heuristiky}

U těchto heuristik platí, že struktura open má \textbf{jen jednu} položku, neprozkoumám typicky všechny možné stavy.

\subsubsection{Best only}

Vrátím \textbf{nejlepší} z okolních řešení. Pokud jsou všechna okolní řešení horší, vrátím prázdný stav. \textbf{Ne}záleží na pořadí procházení sousedů.

\subsubsection{First improvement}

Najdu prvního souseda, který je lepší, pokud žádný není, vrátím prázdný stav. \textbf{Záleží} na pořadí procházení sousedů $\to$ je nutná randomizace.

\subsection{Závěrem}

Většinou provádíme \textbf{kratší}, jednoduché, ale \textbf{rychlé} akce a pokud se to po nějakých MAX\_FLIPS nepovede, provedeme \textbf{složitější} a radikální akci. 
