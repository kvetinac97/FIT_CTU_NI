\section{Experimentální vyhodnocení algoritmu (více instancí)}

\subsection{ECDF}

Mám počet kroků algoritmu (na ose x) a pravděpodobnost, že algoritmus skončil v nejvýše daném kroku. Připomenutí vzorce:

\begin{equation}
    \sum_{k: x_k \leq x}{P(X = x_k)}
\end{equation}

Křivku počítáme jen z úspěšných běhů, ale dělíme ji počtem všech běhů. Vzniká tak tedy pravděpodobnost, že algoritmus \textbf{úspěšně} skončí do kroku $k$.

Average fined par10 je tzv. penalizovaný průměr:

\begin{equation}
    \frac{\sum{k_i} + 10 \cdot l}{n}
\end{equation}

kde $k_i$ je počet kroků v úspěšném běhu, $l$ je limit kroků a $n$ je počet běhů.

\subsection{Porovnávání}

Můžeme porovnávat podle parametrů $\sigma^2$, $\mu$ normálního rozložení; počtu kroků; počtu penalizovaných kroků; nebo také podle xing, winner -- najde se poslední průsečík grafů ECDF a od toho se počítá kdo je nahoře.

\subsection{Jak psát zprávu}

Řekneme, který algoritmus je lepší na těžkých instancích o 20 proměnných (na základě porovnání z měření, co jsme dělali na cvičeních).

\textbf{Material}: standardní implementace gSATu; datové sady ze SATLIBu, uf20-91 (1 000 instancí); primární data (počet iterací, počet splněných klauzulí); 500 iterací max; 1 000 spuštění na 1 instanci; pilotní experiment se 100 spuštěními

\textbf{Results}: co jsme naměřili, k čemu jsme došli (příloha) -- pilotní experiment $\to$ odvozené metriky (četnost úspěchu); celý experiment: metrika -- par10, poslední křížení

\textbf{Discussion}: interpretace dat, zhodnocení významu odpovědi, účinnosti zvolené metody. \textit{(avg fined steps říká, že je větší pravděpodobnost, že B dojde k správnému výsledku, než A, kratší očekávaná doba do úspěšného řešení ;; křížení: při počtu kroků větším než XY je u B větší pravděpodobnost úspěchu)}
