\section{Úvod}

\vspace{12pt}

\subsection{Obsah předmětu}

Témata:
\begin{itemize}
    \item techniky patřící do information retrieval
    \item techniky související s počítačovým viděním
    \item indexace pro rychlejší vyhledávání a namodelování struktury databáze a způsobu dotazování
\end{itemize}

\noindent V přednáškách se budeme zabývat:
\begin{itemize}
    \item platformy pro vyhledávání multimédií (2)
    \item slovníkové metody hledání obrázků (1)
    \item podobnostní vyhledávání v multimédiích (3)
    \item techniky extrakce vlastností (5)
    \item indexace multimédií (2)
\end{itemize}

\subsection{Obsah multimédií}

Biliony fotografií denně, 500 hodin videa se nahraje každou minutu na YouTube. To vše díky rychlému internetu, výpočetní síle, cloudu, principu webu 2.0 (každý je producent) a přesunu lidských aktivit na internet (sociální sítě, banky, e-shopy). Narůstá důležitost mobilních platforem, aplikace jsou používané mnohem častěji než desktop.

Multimédia jsou víc než jeden typ digitálního média (audio, obrázky, video), přesná definice ale neexistuje, je to buzzword. Dá se to všechno ale taky kombinovat -- dokument s obsahem ke čtení, poslechu, vidění, nebo teoreticky i celý web. Většina typů jsou nestrukturované, jedná se ale i o člověkem zpracovaná data.

\subsection{Strukturovaná vs nestrukturovaná data}

Strukturovaná data šla postupně od relačních (SQL) přes objektové, XML, RDF a SPARQL k NoSQL.

Nestrukturovaná data jdou postupně od textu, řetězců, časových řad (série událostí, např. co člověk v čase sleduje), dat ze senzorů, až ke zpracování proudových/cloudových dat.

\subsection{Kategorie producentů dat}

Nejvíce dat produkují \textbf{free-time users} (lidé sdílí fotky z dovolené, z domu, venkovní) -- hodně velká sémantika (lidé, město, příroda, sporty) s komplexními scénami. Tato data nejsou kritická, nemají anotace, jsou pro zábavu.

Pak jsou zde ale data pro \textbf{průmysl a výzkum}, tento obsah je vytvořen profesionály v mnoha velmi úzkých poddoménách (rentgeny, viditelné/neviditelné spektrum, ultrazvuk). Tato data jsou kritická, mají přesné anotace, jsou pro lékařskou diagnózu.

\subsection{Deskriptory multimédií}

\begin{itemize}
    \item \textbf{anotace} -- externí popis s přesnou sémantikou (explicitní: klíčová slova, odkaz, GPS a kontextové: lajky, komentáře, sdílení)
    \item \textbf{deskriptory vlastností na základě obsahu} (extrahované z pixelů, na~nízké úrovni bez sémantiky, např. histogram barev v obrázku)
    \item \textbf{multimodální deskriptory} -- více pohledů \textit{(modalit)} na jednu věc (např. histogram barev a počet hran)
\end{itemize}

Například taková fotografie na Flickru obsahuje vizuální obsah (pixely obráz\-ku), explicitní anotaci (popisek a tagy) a kontextovou anotaci (komentáře ostatních lidí v síti).

Anotace jsou dobré díky vysoké sémantice a jednoduchému vyhledávání, nevýhodou je ale subjektivnost a neúplnost. Generují je dnes také deep learning metody \url{https://cloud.google.com/vision/}

Obsahové vlastnosti jsou vždy přítomné a objektivní, mají ale nízkou sémantiku a hrozí tzv. \textit{semantic gap}.

Anotovat lze v úzké doméně (kritické věci, scan mozku), ale vždy tam musí být člověk, pak jde anotovat i v široké doméně (úroveň objektů -- stůl, počítač, židle, úroveň konceptů -- kancelář, světlé, business).

\subsection{SW architektury pro vyhledávání}

Aplikace zaměřené na \textbf{data} (vyhledávací služby, hosting servery -- vidím fotku hodinek $\to$ jsou to hodinky) nebo na \textbf{služby} (sociální sítě, interaktivní platformy -- vidím fotku hodinek $\to$ zjistím, kde se dají koupit, co to je za značka, kolik stojí).

\subsection{Vyhledávací zdroje}

Hledáme textové dotazy, crawler stáhne všechen dostupný obsah webu, je nutné ale mít model pro samotné vyhledávání -- nejjednodušší je \textbf{metadata-based}, ale u multimédií budeme potřebovat spíše \textbf{content-based}.

\subsection{Hosting servery}

Jedná se o \textbf{úložiště} obsahu a zároveň \textbf{search engine} \textit{(vyhledávací zdroj)}, hledání ale \textbf{pouze} pomocí \textbf{textu} -- překlápí se směrem k sociálním sítím. Existují \textbf{image} hosting servery: alba, galerie -- Flickr, PhotoBucket, ImageShack, a také \textbf{video} hosting servery: YouTube, Dailymotion.

Jsou zde ale také \textbf{stock} servery, které obsahují \textbf{placený} obsah pro profesionální designéry, s \textbf{kontrolovanou} kvalitou obsahu -- Corbis, Getty, iStockPhoto.

Největší služby umožňují také automatizované \textbf{API}, vhodné pro agregaci s~ostatními daty (Google Maps, YouTube, Flickr).

\subsection{Service-centric aplikace}

\textbf{Další krok} vyhledávání multimédií, multimédia nejsou cíl, ale význam, finálním produktem je \textbf{služba}. Multimédium může být \textbf{přímý} konektor \textit{(fotka produktu k nákupu)}, nebo \textbf{nepřímý} \textit{(dva lidé jsou na jedné fotce = můžou mít nějaký vztah)}.

U těchto aplikací se často vyskytují také \textbf{map servery} -- mapa je specifický velký obrázek, kde ale probíhá spíše \textbf{navigace}, než vyhledávání, hledat se dá textově (adresy). Pokročilejší aplikace pak můžou umožňovat zobrazení \textbf{specifických vrstev} -- počasí, turistické informace.

Jednou ze skvělých věcí jsou dnes \textbf{mobilní platformy}, které umožňují snadné vyhledávání na základě multimédií, různé aplikace (Google Lens) a pracuje se i na \textbf{rozšířené realitě} (AR).
