\section{Úvod}

\vspace{12pt}

\subsection{Hodnocení}

5 úkolů, co 2 týdny -- minimálně 20b na zápočet, max. 40 bodů. Písemná zkouška ze tří části po 20 bodech.

\subsection{Plán}

Detaily komunikace, optimalizace výkonu v HTTP, reprezentace dat. Monolitická architektura (aplikační server) vs mikroslužby.

Cloudové služby (Software/Platform/Infrastructure as a service) $\to$ především integrace, aplikační server, ESB (enterprise service bus).

\subsection{Middleware}

Technologie se často mění, architektura se nemění zas tak často. Architektura přidává hodnotu do komunikačního systému a pomáhá při integraci aplikací. Jedná se o součást webu 2.0. Svět internetu jde rychle, ale nasazení do praxe bývá pomalejší.

\vspace{6pt}

Co může být middleware?

\begin{itemize}
    \item identity access management (IdAM)
    \item messaging-oriented middleware (MOM) -- \textit{synchronní, asynchronní komunikace}
    \item protokol (např. SOAP, ale obecně cokoliv pro komunikaci)
    \item škálovatelnost, load-balancing (nginx)
    \item monitoring, logování dat, firewall
    \item stejně jako se snažíme o přepoužitelný kód, tak stejně se snažíme i o přepoužitelné služby
    \item chceme efektivní aplikace, šetřit náklady
\end{itemize}
