\section{Řešení}

Řešení je na úrovni jedince $\to$ rodiny $\to$ společnosti $\to$ státu $\to$ EU. Existuje mnoho \textbf{oblastí}, ve kterých lze tento problém řešit \textit{(kritické myšlení, nešířit dezinformační články v novinách, hledat technická řešení, uspořádat uměleckou výstavu)}.

\subsection{Jedinec}

Na úrovni jedince se jedná o \textbf{kritické myšlení}. Člověk může mít různou hloubku vs šířku znalostí. \textbf{I} \textit{(jsem specialista na jedno téma)}, \textbf{T} \textit{(jsem specialista na 1 téma + mám nějaké povědomí o zbytku)}, $\pi$ \textit{(mám 2 specializace + povědomí o zbytku)}, \textbf{M} \textit{(více specializací a povědomí o zbytku)}.

To, jací jsme, je určeno: schopností \textbf{přijímat} informace $\to$ schopnosti \textbf{filtrovat} informace $\to$ schopnosti kreativního myšlení (schopnost \textbf{vytvářet} informace). \textit{(Kromě informací pak máme ještě laskavost a tělesnou stránku)}.

\subsubsection*{Kritické myšlení}

Kritické myšlení = nástroj proti hlouposti, pečlivá aplikace \textbf{rozumu} k rozhodnutí, zdali je tvrzení \textbf{pravdivé}. Náš mozek funguje ve dvou módech: \textbf{rychlý} \textit{(emoce, reflexy, afekty)} a \textbf{pomalý} \textit{(to je to kritické myšlení)}.

\subsubsection*{Kognitivní zkratky vol. 2}

Už jsme si říkali, že máme nějaké kognitivní zkratky. Když je známe, jsme schopni \textbf{rozpoznat}, že to je ta \textbf{zkratka} a ne náš rozumný \textbf{úsudek}.

Např. \textbf{in-group bias}: když jsme v homogenní skupině, zkresluje to naše úsudky \textit{(jak je možné, že někdo mohl volit tuto stranu)} nebo \textbf{confirmation bias}: tendence věřit tomu, co nám vyhovuje.

\textbf{Backfire effect}: když je názor pro nás tak důležitý, že je součást naší identity a někdo nám je začne vyvracet, ještě nás to více utvrdí \textit{(pozor, které názory jsou součástí naší identity)}.

\textbf{Informační obezita}: mozek je neuroplastický -- vše, čemu se vystavujeme, se obtiskne, klíčová je i \textbf{vědomost} \textit{(vybírejme si témata, která bezprostředně ovlivňují náš život, co se týkají našeho oboru a celospolečenská témata -- jaká média: \url{https://www.nfnz.cz/rating-medii/})} 

\textbf{Occamova břitva}: pokud existuje nějaké jednodušší řešení, tak bude spíše blíže pravdě to jednodušší řešení.

Technika \textbf{ZZZ}: Když přijde názor, měli bychom zkusit ho: \textbf{z}pochybnit $\to$ \textbf{z}vážit a vytvořit \textbf{z}ávěr.

\textbf{Metakognice}: schopnost poodstoupit a nedívat se na to, co si myslíme, ale na to, \textbf{jak} jsme k tomu došli.

\textbf{Learning journal}: obsah \textit{(co mě zaujalo)} $\to$ emoce \textit{(jaké emoce to způsobilo)} $\to$ otázka \textit{(jaké otázky to probudilo)} $\to$ akce \textit{(co na základě toho udělám)}.

\subsection{Rodiny a společnosti}

Za jakých podmínek jsme ochotni zapnout \textbf{pomalé} myšlení? První krok je \textbf{motivace} -- vzdělávat násilím nelze. Druhým krokem je \textbf{čas} -- trvá to nějakou dobu, ne každý moment je vhodný pro diskuzi. Třetí složkou jsou \textbf{dovednosti} -- jinak se budeme bavit s dítětem, nebo dospělým.

K tomuto můžeme jít dvěma cestami -- \textbf{centrální cesta} je logické myšlení (pomalé), \textbf{periferní cesta} je nárazové, řízené strachem a touhou (rychlé).

\textit{Pokud máme při diskuzi problém s časem, udělejme si ho. Pokud motivace, sdělme se s tím, že máme problém. Dovednosti = dá se naučit. ad. kniha Jak vést (zdánlivě) nemožné rozhovory: Naslouchat, Otázky, Společné zážitky.}

\subsection{Skupiny}

\textbf{Barack Obama}: nemůžeme se vypořádat s výzvami 21. století byrokratickými přístupy 20. století \textit{(demokratický systém je pomalý)} $\to$ musíme na to jít \textbf{zeshora} \textit{(od EU, státu, společnosti)}.

Obecně pro \textbf{informační platformy} platí, že je lepší, aby \textbf{nebyly} centralizované (Meta -- Facebook + Instagram, Google, App Store) $\to$ jedním z řešení může být nadnárodní regulace (EU) -- \textbf{Antitrust}.

V tomto směru je možná \textbf{budoucnost} decentralizovaných sítí, ještě lepší distribuovaných sítí. \textit{(Jak jsme říkali, většina obsahu je moderována jen v anglicky mluvícím světě)}. Stejně jako kdysi \textbf{nešlo} přejít mezi telefonními operátory, tak dneska nejde přejít z Facebooku na XY.

Jsou zde snahy o \textbf{interoperabilitu} na úrovni posting, messaging a friend finding mezi sociálními sítěmi: \textbf{ACCESS Act}.

Na úrovni EU existuje \textbf{Digital Markets Act} -- namířen proti velkým firmám, pod hrozbou \textbf{sankcí} (10 \% global revenue): interoperabilita \textit{(pouze)} \textbf{messaging} aplikací, právo odinstalovat aplikace \textit{(Chrome na Androidu, Safari na iPhone)}, zákaz self-referencingu, přístup k analytickým datům.

EU -- \textbf{Digital Services Act}: boj s ilegálními produkty, službami \textit{(tvrdé drogy, zbraně)}, dohledatelnost prodejců v online marketplacům, možnost odvolání proti banu, průhlednost v použitých algoritmech, snaha předvídat rizika různých akcí, lepší přístup pro výzkumníky.

\textit{(Například v Číně je zde regulace pro děti do 14 let -- TikTok pro děti je max 40min denně, offline 22-6, 33.3 \% zábava, 33.3 \% edukace, 33.3 \% propaganda)}

\textbf{DAO}s: máme model sběru dat $\to$ jejich ukládání $\to$ cílení a personalizace, je otázka, zda při dělení nechat \textbf{stejný} model na jednotlivé platformy, nebo mít spíše jednu na sběr dat, druhou na ukládání a třetí na cílení. \textit{(Je to ale bezpečné, aby měla firma přístup k našemu genomu, jen za účelem zisku?)} $\to$ \textbf{Data Trust} organizace spravující data: \textit{(držely by ho Decentralizované Anonymní Organizace.?)}

Už máme zákony na regulaci násilí proti skupině osob, jednotlivci, vyhrožování, pronásledování, rasismus\ldots \space jen je aplikovat. \textbf{Humor} se šíří ještě rychleji než nenávist \textit{(a ta rychleji než pravda)}.
