\section*{Úvod}

\vspace{12pt}

\subsection*{Hodnocení}

Přednášky každý týden + cvičení co dva týdny. 60 \% známky bude z obhajoby semestrálního projektu, 40 \% ze zkoušky. Semestrální projekt bude ve čtyřčlenných týmech, volba tématu 11. října, 2 checkpointy (definice problému, zamýšlené řešení) + finální obhajoba. Zkouška bude formou písemného testu.

\subsection*{Náplň předmětu}

\textbf{Dezinformace} je lživá informace vytvořená se záměrem matení veřejnosti. Ten pojem byl vymyšlen Stalinem ve dvacátých letech 20. století a byl aktivně zapsán do sovětského slovníku.

\begin{quote}
    \textquote{Jsme v informační válce, a ta je předstupněm války reálné.}
\end{quote}

Cílem dezinformátorů je vyvolat konflikty a nedůvěru na všechny strany. Máme moderátora, pak trolly, kteří vytváří a distribuují obsah na základě těchto pokynů do cílových skupin, aby si ho začaly šířit. Mezi důchodci to jsou řetězové e-maily, celkově to můžou být sociální sítě a masová média. Ideální pro šíření jsou emoce, především vztek, ale ještě silnější je vtip.

\begin{quote}
    \textquote{Máme pravěké emoce, středověké instituce a Božskou technologii.}
\end{quote}

Změna jen jedné části technologie v rámci procesu vytváří \textbf{productivity paradox}, kde pokud se nezmění celý ten proces, tak to je ještě méně efektivní. Internet a sociální média se dají přirovnat k vynálezu knihtisku, kdy najednou přechází autorita od církve k lidem, začínají velké revoluce. My jsme v téhle době nestability, kdy roste populismus a bude to nejen o technologiích, ale i o lidech.

\textbf{Epistemická krize} je útok na sdílené poznání světa. Dnes už je umělá inteligence schopná generovat obrázky, u kterých se těžko poznává, jestli jsou reálné, nebo umělé. Člověk má pak problém poznávat, a zjistit, co je pravda, a co ne.

\begin{quote}
    \textquote{Manipulace je ovlivňování někoho jiného pro můj prospěch.}
\end{quote}

Manipulace je hodně o opakování stejných informací, o dopaminu. Existuje více spekter manipulací:

\begin{itemize}
    \item propaganda \textit{(zisk moci)}
    \item komerce \textit{(zisk finanční)}
    \item reputační \textit{(poškození reputace)}
    \item well-being \textit{(fitness aplikace, která motivuje k cvičení)}
\end{itemize}
