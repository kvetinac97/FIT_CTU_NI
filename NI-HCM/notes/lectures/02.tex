\section{Historie Globálního Kognitivního Systému}

\vspace{8pt}

\noindent \textbf{Marshall McLuhan} (1964): jakákoliv technologie je rozšířením člověka, technologie = médium, něco nesou, nejsou neutrální

\vspace{4pt}
\noindent \textbf{Technologický determinismus}: média mění chování, myšlení, cítění a sociální uspořádání \textit{(od společnosti až na úroveň jednotlivce)}

\vspace{8pt}
\noindent \textbf{Tetráda efektů médií}:
\begin{itemize}
    \item \textbf{enhance}: co to obohacuje, přináší nového?
    \item \textbf{obsolete}: co to mění, až to není důležitého?
    \item \textbf{retrieve}: znovu obnovuje to, co už nebylo důležité?
    \item \textbf{reverse}: když to posuneme do extrémů, co se z toho může stát?
\end{itemize}

\noindent Nejprve jsme transformovali \textbf{materiál} (kamenné, bronzové nástroje \ldots), pak \textbf{energii} (voda, pára, elektřina), nyní transformujeme \textbf{informaci} (nejprve komunikace, nyní znalost).

\subsection{Slovo}

Vše to začalo \textbf{slovem} \textit{(viz. Genesis 1,1)} -- informace se přenášely nejprve slovně. Existovalo sice písmo, ale bylo složité ho reprodukovat. I v Antice byla rétorika součástí základního trivia.

\subsection{Knihtisk}

Pak přišel \textbf{knihtisk} a kultura získala přístup ke gramotnosti. V období francouzské revoluce přišla první adopce \blockquote{novin} s informacemi o politice, což udržovalo revoluci v proudu \textit{(první populismus a manipulace médii)}. V roce 1830 se objevuje v Americe první bulvár, s tím i paparazzi a první diskuze o právu na soukromí.

Už paralelně s knihtiskem vznikaly \textbf{první konspirace} -- Protokoly Sionských Mudrců (1903) -- prazáklad antisemitských konspirací.

\subsection{Rádio}

V roce 1895 přišli bratři Lumierové a film, 1897 první \textbf{rádio} (Marconi), v roce 1922 vzniklo BBC a první \textbf{veřejnoprávní médium} BBC Charter (1927) s cílem produkce obsahu vysoké kvality, podpory kreativního průmyslu a propagace kultury Spojeného království. První vysílání Českého rozhlasu bylo v roce 1923.

Podle této \textbf{tetrády efektů} médií rádio: obohacuje (novinky na velkou vzdálenost), upozaďuje film, obnovuje mluvené slovo.

V roce 1938 se neúmyslně podařilo zmanipulovat Američany rozhlasovou hrou o mimozemšťanech tak, že se začali schovávat do krytů se zásobami jídla a brokovnicemi.

Nacisté rádio také použili k propagandě v roce 1933 -- státní médium, zneužití. \textit{(U BBC se jedná pouze o hodnoty, Charter, které má televize reprezentovat)}

\subsection{Televize}

V roce 1926 vzniká první televize, 1928 barevná televize \textit{(přesto ještě v 60. letech byla televize černobílá)} a od roku 1936 začíná pravidelné vysílání BBC.

K propagandě byl za války a komunismu používán také týdeník, který běžel v kině před filmem.

Česká televize začíná vysílat v roce 1953 a jedná se o státní televizi, řízenou komunisty plnou propagandy -- společnost je \textbf{manipulována}.

Mezitím v Americe proběhla v roce 1960 první prezidentská debata v historii, tam se Nixon ztrapnil oproti mladému, dynamickému Kennedymu, což víceméně \textbf{předurčilo} výsledek prezidentských voleb.

\subsection{Informační věk}

Informace \textit{(in formation -- seřazené byty)}

\vspace{4pt}
\noindent V roce 1968 výzkumné středisko udělalo \textbf{první demo} -- myš, klávesnice, souborový systém\ldots, trvalo ale další desítky let, než se to uchytilo.

V roce 1981 začíná MTV -- hudební televize 24/7, opět to patří do \textbf{tetrády}: obohacuje hudbu o vizuální informaci, upozaďuje rádio, znovu obnovuje tanec a performance, když to přeženeme, mění to schopnost vnímat dlouhý typ informací.

\vspace{4pt}
\noindent \textbf{Jean Baudrillard: Simulacra \& simulation} -- žijeme v simulaci, 4 fáze:

$\to$ odraz: fotka ryby

$\to$ maska: upečená ryba, dochází k distortion

$\to$ iluze: rybí prsty, symbol ryby

$\to$ čistá simulacra: Oreo ve tvaru ryby

\vspace{4pt}
\noindent \textbf{Noam Chomsky -- Manufacturing Consent}: \blockquote{Chytrý způsob, jak udržet lidi pasivní a poslušné je přísně omezit spektrum přijatelných možností a vytvořit v jejich rámci velmi živou debatu}

\vspace{8pt}
\noindent Identifikuje \textbf{5 filtrů} komerčních médií:

\begin{enumerate}
    \item Vlastníci médií
    \item Business model reklamy
    \item Spoléhání se na kolaboraci s mocí \textit{(newsworthy)}
    \item Disciplinování médií \textit{(žaloby, dohled -- napadání veřejnoprávních médií)}
    \item Hledání vnějšího nepřítele
\end{enumerate}

\noindent U nás máme jak tyto komerční média, tak \textbf{veřejnoprávní média} (nejsou státní, státní jsou pouze \textbf{rady} nominované státem):

\begin{itemize}
    \item snaha o \textbf{neutralitu}, nezávislost na politických stranách
    \item \textbf{objektivní}, poskytnout více zdrojů
    \item \textbf{vyváženost} -- majoritní i menšinové názory
    \item \textbf{etický kodex} média a redaktora (ředitelská i redaktorská)
\end{itemize}

\subsection{Internet}

Koncem 80. let se začíná rozmáhat Internet (1980) a v roce 1989 přichází Tim Berners Lee a World Wide Web -- URI, HTTP a HTML.

\vspace{4pt}
\noindent 1994: přichází první cookie soubory, fact checkingová stránka Urban Legends Reference Pages (dnes Snopes, user generated content) $\to$ 1995: Cragslist (bazar, první stránka, kde lidé začínají vytvářet obsah).

\vspace{4pt}
\noindent 1994: První reklamní banner na HotWired, pak se rozvíjí Ad Servers \textit{(cílení reklamy, performance tracking)} $\to$ 2000: Google AdWords $\to$ 2002: AdBlock, Pay-per-click \textit{(předtím Pay per view)}

\vspace{4pt}
\noindent 1996: \textbf{Section 230} -- první regulace internetu: provozovatel služby není zodpovědný za věci, které tam sdílí jiní lidé \textit{(sociální sítě nejsou zodpovědné za dezinformace, "jen poskytují infrastrukturu")}.

\vspace{4pt}
\noindent 1997: PageRank $\to$ 1998 \textbf{Google} -- v miliardách webových stránek jsme schopni se dobře vyznat.

\vspace{4pt}
\noindent McLuhanova tetráda: \textbf{obohacuje} o decentralizaci, velká rychlost, elektronická komunikace, self-publishing; \textbf{znovu obnovuje} psaní, sociální bubliny, anarchie, chaos; \textbf{upozaďuje} cestování, vzdálenost, hranice, fyzické obchody, centrální cenzuru; \textbf{pokud se to přežene}, všechno je datově řízené, máme přebytek informací \ldots

\vspace{4pt}
\noindent \textbf{Podcasty} (Apple, 2001) -- distribuovaná síť šíření obsahu -- podcaster vytvoří RSS feed, z těch si to berou streamovací služby \textit{(Podcasts, Spotify\ldots)}, které to zpřístupní posluchačům.

\vspace{4pt}
\noindent 2004 přichází \textbf{Web 2.0} $\to$ read-write web, Tim O'Reilly $\to$ 2005 YouTube $\to$ 2007 iPhone \textit{(změna paradigmatu / paradigm shift)}

\subsection{Sociální sítě}

Začínají v roce 2003 s MySpace a LinkedIn, 2004 se připojuje Facebook, 2010 Instagram, 2011 SnapChat, 2017 TikTok \ldots

\vspace{4pt}
\noindent McLuhanova tetráda: \textbf{obohacuje} o spojení s nelokálními skupinami; \textbf{znovu obnovuje} příslušnost ke skupině, polarizace; \textbf{upozaďuje} socializaci, face to face; \textbf{pokud se to přežene}, vzniká manipulace, sociální izolace \ldots

\vspace{4pt}
\noindent \textbf{Bias} (např. selection bias s letadly) je schopnost naučení AI na základě dat a velmi záleží na použitém modelu pro strojové učení. Podobný algoritmus je ale použit pro výběr příspěvků na sociálních sítích -- pozor na bias!
