\section{Digitální manipulace}

Specifikum \textbf{digitální manipulace} je zaměření na množství, opakování a kontext. Známe zde 4 oblasti: propaganda, komerce, reputace, well-being.

\subsection{Epidemiologická triáda}

Fungování lidského mozku souvisí s epidemiologickou triádou -- \textbf{agent} \textit{(virus)}, který napadá \textbf{host}a \textit{(člověk)} a šíří se \textbf{prostředím}. To se dá aplikovat i na on-line prostředí.

\vspace{4pt}
\noindent U digitálních manipulací jsou prostředím on-line \textbf{algoritmy}, agentem je \textbf{obsah} a zkušenosti a hostem je lidský \textbf{mozek}. Účastníkem tohoto všeho jsou pak ještě navíc aktéři \textit{(troll, propagátor reklamy)}.

\subsection{Lidský mozek}

Naše smysly jsou schopny přijímat \textbf{desítky miliónů} bitů za sekundu, ale racionální kapacita \textit{(co jsme schopni zpracovat)} mozku jsou \textbf{stovky bitů} za sekundu. Z toho většina informací je přijímána \textbf{očima}. Všechny zpracované informace jsou ovlivněny i hlubinnými procesy v mozku \textit{(vidíme patterny tam, kde nejsou)}.

\vspace{4pt}
\noindent Přenos informací v mozku je určen z části elektrosignály, ale také \textbf{neurotransmitery} -- vylučováním chemických látek, hormonů, jako jsou adrenalin, \textbf{dopamin}, endorfiny apod.

\vspace{4pt}
\noindent \textbf{Neuroplasticita}: schopnost mozku se měnit v čase, neuronů se propojovat

$\to$ strukturální: fyzická změna na základě učení

$\to$ funkční: přesun funkcí z poškozených míst na ta nepoškozená

\vspace{4pt}
\noindent \textit{(Do 2 let nabírá mozek na objemu, mezi 2 a 10 se prořezávají cesty v mozku, od 10 do 20 se buduje sociální inteligence a indentita, v 25. roce se to už značně zpomaluje.)}

\subsubsection*{Mere Exposure Effect}

\textbf{Opakováním} informace se začíná člověku věc \textbf{líbit} -- opakování písničky v rádiu vám ji dostane do mozku, proto fungují billboardy na Times square, awareness programy\ldots

\subsubsection*{Limbický systém}

V mozku máme různé cesty -- jednu pro paměť, pro pohyb, ale také hypothalamus, který vylučuje dopamin. Ten putuje po \textbf{mesolimbické cestě} a ovlivňuje naše plánování, rozhodování a kreativitu. \textit{(důležité pro manipulaci)}

\subsubsection*{Flow State}

Stav, kdy se do něčeho ponoříte tak, že přestanete vnímat fyzikální potřeby, hodně vás to baví, nevnímáte čas. Do tohoto stavu se dostanete na základě důležitosti, obtížnosti výzvy a tom, jaké si člověk myslí, že má schopnosti \textit{(jinak úzkost/strach vs nuda)}.

\vspace{4pt}
\noindent Během flow vylučuje mozek \textbf{dopamin} -- soustředění, poznávání nových patternů; \textbf{endorfin} -- radost; \textbf{serotonin} -- after-glow efekt, uspokojující pocit po dohrání hry.

\subsubsection*{Gamifikace}

Zapojení herních prvků do činností, které nejsou hra.

\vspace{4pt}
\noindent Kolečko: $\to$ Spouštěč $\to$ Akce $\to$ Odměna $\to$ Osobní investice

\vspace{4pt}
\noindent Jedním z dobrých příkladů jsou \textbf{badge}, na sociálních sítích to může být \textbf{like}, vzniká tam flow.

\subsubsection*{Thinking Fast \& Slow}

V mozku se dějí dva systémy přemýšlení: \textbf{podvědomí} (rychlé, náhodné, komplexní) a \textbf{vědomí} (pomalé, uspořádané, jednoduché).

\subsubsection*{Kognitivní zkratky}

Souvisí s tím, co máme dělat: potřebujeme se rozhodnout rychle, co si máme zapamatovat, nemáme dost informací, informací je až moc.

\vspace{4pt}
\noindent Např. \textbf{confirmation bias}: máme více rozdílných informací, kterou z nich si zapamatovat? $\to$ zapamatujeme si to, co se nám líbí nejvíce

\subsubsection*{Behaviorální modely}

Na základě kognitivních zkratek se dá lidi klasifikovat do behaviorálních modelů a kvantifikovat lidi. OCEAN x Mayers-Briggs (E/I, S/N, T/F, J/P) x DISC\ldots

\subsection{Komerční manipulace}

Marketingová kampaň může mít různé cíle -- \textbf{upozornit na produkt}, přilákání uživatelů, zvýšení prodejů, nové aktualizace, vybudovat komunitu\ldots

\vspace{4pt}
\noindent Jedním z častých způsobů upozornění na nový produkt / službu jsou \textbf{billboardy}, které velmi dobře fungují pomocí neuroplasticity mozku, a tím se dostaneme do \textbf{povědomí}.

\vspace{4pt}
\noindent Zvláště trh s akciemi je hodně citlivý -- boost akcií nějaké staré hry, skok Bitcoinu na základě oznámení Elona Muska o zastavení možnosti nákupu Tesly za Bitcoin.

\vspace{4pt}
\noindent \textbf{Data brokeři}: nakupují data z Facebooku, z věrnostních kartiček v supermarketech

\subsection{Reputace}

Reputace je o \textbf{důvěře}, statutu a PR. Cílem reputační manipulace je \textbf{očernit} ostatní a poškodit jim reputaci. Dá se k tomu využít boty i influencery.

\vspace{4pt}
\noindent Trollům se tak podařilo dehonestovat šéfku hnutí protestu žen, protože měla iránské kořeny.

\subsection{Well-being}

Snaha o poškození dobrého pocitu, psychického zdraví / stavu. Patří zde také \textbf{phishing}, snahou je získat data ostatních, aby je pak šlo vydírat.

\vspace{4pt}
\noindent Příkladem byly i \textbf{lootboxy} v dětských hrách, které mezi levely zapojují mechaniku výherního automatu.

\subsection{Propaganda}

Propaganda vznikla v katolické církvi za účelem šíření víry. V propagandě jde o manipulaci lidského myšlení za účelem \textbf{získání moci}. Patří zde ale také strategická komunikace, politický marketing, populismus a dezinformace.

\subsubsection*{Strategická komunikace}

hodnoty: za čím stojíme

$\to$ vize: kam jdeme, čeho chceme dosáhnout

$\to$ mise: co děláme, pro koho

$\to$ strategie: jak budeme postupovat

$\to$ taktika: co musíme udělat

\vspace{6pt}
\noindent \textbf{Politický marketing a populismus}: Hranice je těsná -- v politickém marketingu jde o manipulaci k \textbf{reálné akci} za účelem vaší \textbf{vize}, populismus je \textbf{reakce na lidi}, je to bez vize.

\vspace{6pt}
\noindent \textbf{Propaganda}: Už skutečné zneužívání státních médií a šíření lživých informací. Šíření falešné reality.

\vspace{6pt}
\noindent \textbf{Dezinformace}: Matení veřejnosti, viz. příště.

\subsection{Dynamika skupin}

Žijeme v době digitálních kultů (sekt) -- probíhá radikalizace určité skupiny lidí, dostávají se do izolace a naslouchají lídrovi \textit{(kult Donald Trump)}. Kult \textbf{QAnon} = konspirační sekta \textit{(svět ovládá temná struktura, unáší děti a zabíjí je)}.

\vspace{4pt}
\noindent Digitální kulty vznikly z \textbf{L}ive \textbf{A}ction \textbf{R}ole-\textbf{P}laying her, vznikla zde Q sekce, amplifikace ruskými botnety $\to$ napadení kapitolu \textit{(= vrchol manipulace, radikalizace, bude potřeba ozbrojený převrat; temná síla = ještěři / židé)}
