\section{Dezinformace}

U \textbf{obsahu} vzniká \textbf{memetika} a \textbf{prostředí} vytváří \textbf{zkušenosti}. Fyzický svět je neustále více obohacován $\to$ Augmented Reality $\to$ Mixed Reality $\to$ Augmented Virtuality $\to$ Virtual Reality.

\subsection{Prostředí}

\textbf{Automatizované}: e-shopy, hry, virtuální světy; \textbf{sociální}: náhodné, lidmi řízené -- v každém případě toto prostředí je řízené algoritmem -- kdo vládne, proč, s jakými pravidly?

\vspace{4pt}
\noindent \textbf{KPI}: metriky, které určují výkonnost systému \textit{(key performance indicator)} -- na internetu se používá \textbf{Conversion Funnel}: fáze, kterými lidé prochází \textit{(Discovery [objevení] $\to$ Interest [zájem] $\to$ Appraisal [výzkum] $\to$ Confirmation [potvrzení] $\to$ Conversion [nákup])}, všechny tyto fáze jsou měřitelné přes KPI.

\vspace{4pt}
\noindent Na e-shopech se dá použít AB vs \textbf{Multivariantní testování}: uživateli se v případě zobrazení produktu bez nákupu zkusí ukázat tlačítko Koupit na jiných pozicích.

\vspace{4pt}
\noindent \textbf{Nudge}: politika šťouchání -- snaha ovlivnit chování lidí předvídatelným způsobem, dá se využít k dobru \textit{(dobrovolné darování orgánů po smrti)}

\vspace{4pt}
\noindent \textbf{Dark patterns} ve hrách: návrh rozhraní takového, aby byli uživatelé motivováni k chování, které by jinak normálně neudělali $\to$ \textbf{pay to skip} (přeskočení reklamy), \textbf{pay to win} (whale -- člověk, co si vše koupí, aby vyhrál), \textbf{artificial scarcity} (omezený produkt), \textbf{accidental purchases} (snadný nákup), \textbf{loot boxes} (automaty), \textbf{anchoring tricks} (drahé produkty vedle levných).

\vspace{4pt}
\noindent \textbf{Dark patterns} v e-commerce: musíme vynaložit větší úsilí na odhlášení z newsletteru, opt-out u předplatných, \textbf{urgency messaging} (na tento produkt se dívá 10 lidí a je poslední kus), \textbf{drip pricing} (začne s nízkou cenou u objednávání hotelu, na konci + rezervační poplatky, člověk už to radši objedná), \textbf{peer pressure} (závisí na lidech kolem nás, tendence k davovému chování $\to$ když ve feedu přidali \blockquote{Josef taky hlasoval}.)

\vspace{4pt}
\noindent Na tady toto existuje přímo i \textbf{social bias} (porovnáváme se s ostatními), \textbf{default bias} (tíhneme k věcem, které známe) a \textbf{loss aversion} (když už jsme do něčeho investovali čas, máme menší tendenci to vzdát).

\subsection{Memetika}

Stejně jako v evoluční teorii je variace + selekce + dědičnost předpokladem evoluce, tak stejně i \textbf{meme} se šíří podobně jako sobecký gen. \textbf{Meme} = jednotka kulturní informace šířená imitací \textit{(sobecky se šíří)}, obecně meme je něco, co se dá šířit. Přenáší se komunikačními kanály, \textbf{obsahem} jsou instrukce nebo nudges. \textbf{Forma} je jednoduchá, úderná, vyvolává emoce.

\vspace{4pt}
\noindent \textbf{Memeplex}: sada memů, které reprezentují individualitu (kombinace politických, náboženských názorů, sociálních postojů, tradicí, zvyků, paradigmat).

\subsection{Analýza obsahu}

\vspace{4pt}
\noindent \textbf{Deskriptivní}: analýza obsahu z různých perspektiv (NLP -- témata, vztahy, pravdivost, bias, sentimenty)

\vspace{4pt}
\noindent \textbf{Inferentiální}: charakteristika komunity, která obsah vytváří / konzumuje, co je zajímá, sociologický pohled

\vspace{4pt}
\noindent \textbf{Predikční}: sleduje křivku práce s daty, používána online algoritmy pro prioritizaci obsahu, na který s největší pravděpodobností klikne cílové publikum.

\subsection{Gerasimova doktrína}

Drtivá většina toho, co je obsahem \textbf{moderní války}, je ne-válečné, je to o snaze ovlivnit politiky jiných států, mít sféry vlivu, o ekonomických sankcích.

\vspace{4pt}
\noindent Válka je dnes: \textbf{hybridní} (ekonomické sankce, válečné operace), \textbf{kognitivní} (vláda nad myslí), \textbf{informační} (ovládání médií), \textbf{cyber} (DDOS útoky, ovlivňovací operace) $\to$ i vojáci zachází s těmito psychologickými operacemi

% ============================

\section{Algoritmy sociálních sítí}

Facebook má patent na zjištění sociální úrovně z fotek \textit{(fotka vedle mercedesu / škodovky, v Gucci / HM)}, nepoužívá ho ale, protože koupit si tyto informace od data brokerů je snadnější.

\subsection{Dezinformace}

Misinformace může být pouze \textbf{omyl}, jde obecně o lež, malinformace je \textbf{úmyslné} poškození za cílem poškození reputace. Dezinformace je průnikem obou = lživá za účelem poškození.

\subsection{Řetěz dezinformací}
Nalézt trhliny [\textbf{recon}] (analýza cílovky, vytvoření plánu)

\vspace{2pt}
\noindent $\to$ aktivizace [\textbf{build}] (vytvoření infrastruktury, aktivace lidí)

\vspace{2pt}
\noindent $\to$ spuštění kampaně [\textbf{seed}] (puštění mezi lidi)

\vspace{2pt}
\noindent $\to$ rozmnožení [\textbf{copy}] (šíření pomocí \blockquote{užitečných idiotů})

\vspace{2pt}
\noindent $\to$ zesílení [\textbf{amplify}]

\vspace{2pt}
\noindent $\to$ manipulace [\textbf{control}] (vytvoření konfliktu)

\vspace{2pt}
\noindent $\to$ sklizeň [\textbf{effect}] (akce, neexistuje žádná pravda).

\subsection{Prostředí -- sociální sítě}

Na vstupu do sociální sítě je nějaký \textbf{filtr}, ten může být opět biased kvůli machine learning, pak se data uloží, a některá se \textbf{doporučí} koncovým uživatelům.

\subsubsection*{Variable Reinforcement Schedule}

Pokaždé nevyhrajete, jen jednou za čas, ale hrajete pořád, dokud tu výhru nedostanete. \textit{(používá se v herních automatech, u gamblingu)}

\vspace{4pt}
\noindent Tak stejně nahráváme příspěvky, abychom dostali dopaminovou injekci a likes. Algoritmy se tedy snaží o maximalizaci počtu příspěvků $\to$ stejně jako v automatech nedostaneme hned všechny likes, ale \textbf{postupně} v dávkách.

\vspace{4pt}
\noindent Služby samozřejmě sbírají ještě více dat, aby měly co prodávat nabízejícím službám (taková pravděpodobnost, že člověk klikne na reklamu).

\subsubsection*{Doporučovací systémy}

\textbf{Item-based}: koukám na pračku, dostanu doporučení na podobný výrobek. Koukám na komedii, dostanu další komedie.

\vspace{4pt}
\noindent \textbf{Kolaborativní filtrování}: vám podobným lidem se také líbilo. Koukám na pračku, lidé se kromě pračky dívají i na sušičku, tak dostanu doporučení na sušičku.

\subsubsection*{Pojmy}

\textbf{Network effect}: Hodnota sítě je přímo úměrná mocnině počtu lidí, kteří na ni jsou \textit{(čím víc uživatelů, tím více možných interakcí\ldots)}

\vspace{4pt}
\noindent \textbf{Triangle closing}: Základní princip \textbf{PYMK} (people you may know) = lidé s podobnými zájmy jsou v síti blízko $\to$ pokud A zná B a C, pak by se B a C měli také poznat = algoritmus \textbf{zavírání trojúhelníku}.

\vspace{4pt}
\noindent \textbf{Metcalfův zákon}: Dynamika sítě je určena především počtem propojení mezi uživateli \textit{(čím více, tím lepší zkušenost s danou sociální sítí)}.

\vspace{4pt}
\noindent \textbf{Granovetterovy vazby}: každý z nás má některá \textbf{silná} spojení \textit{(rodina, přátelé)}, ale také \textbf{slabá} propojení \textit{(velmi důležitá, např. hledání práce, najde lidi z jiného prostředí}

\vspace{4pt}
\noindent Jako lidé navazujeme vztahy primárně podle \textbf{společných} zájmů, aby nás lidé podpořili, mohli jsme mu \textbf{důvěřovat}, na základě síly \textit{(využíváme něčí moc)}, výměna znalostí, identita, status a romance.

\vspace{4pt}
\noindent Intenzita propojení: 5 blízkých přátel, 15 dobrých, 50 přátel, 500 známých, 1 500 povědomí. Ke každému na světě se dostaneme přes \textbf{méně} než 6 propojení.

\vspace{4pt}
\noindent Typicky 1 \% uživatelů \textbf{vytváří} obsah, okolo 10 \% ho \textbf{šíří} \textit{(vytváří thready, odpovídá, dá obsah do skupiny)} a všichni ho \textbf{konzumují} $\to$ jen málo lidí opravdu vytváří dezinformace.

\vspace{4pt}
\noindent \textbf{Astroturfing}: uměle vyvolané hnutí, které se tváří přirozeně

\vspace{4pt}
\noindent \textbf{Filter Bubble}: dostáváme se do bubliny tím, že nám algoritmy filtrují obsah. \textbf{Epistemic Bubble}: bublina lidí s jednotným světonázorem $\to$ \textbf{Echo Komora}: Epistemic Bubble spojená s nedůvěrou k ostatním. Problém není v algoritmu \textit{(a jednolitém obsahu)}, ale v tom, že se sami \textbf{navzájem} utvrzujeme.

\subsubsection*{Influence Continuum}

Snaha \textbf{zmanipulovat} člověka je popsána v BITE modelu \textit{(chování, informace, myšlení a emoce)}.

\textbf{Zdravá} je autentičnost, milost, soucit, uvědomění, kritické myšlení \textbf{vs} identita kultu, podmíněná láska, nenávist, doktrína, závislost.

Zdravý \textbf{vůdce} zná svá omezení, pomáhá jednotlivcům, je důvěryhodný \textbf{vs} narcista, hladový po moci, chce absolutní autoritu.

\vspace{4pt}
\noindent Destruktivní \textbf{kult} tedy obsahuje TOP vedoucí \textit{(jednotlivci)}, \textbf{amplifikátory} a ve výsledku \textbf{oběti} \textit{(kteří už ani neví, co jde od vedoucích a co od amplifikátorů, postupná radikalizace stávajících hashtagů)}.

\vspace{4pt}
\noindent Kult \textbf{kryptoměn}: zničit stát a nahradit ho blockchainem, kult Elona: Tesla nebude přijímat Bitcoiny $\to$ obrovský pokles hodnoty, \textbf{vnější nepřítel}: zlá vláda, centrální banky; \textbf{utopické} narativy.
